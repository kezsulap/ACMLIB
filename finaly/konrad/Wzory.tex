%%%%%%%%%%%%%%%%%%%%%%%%%%%%%%%%%%%%%%%%%%%%%%%%%%%%%%%%%%%%%%%%%%%%%%%%%%%%%%%%

\usepackage[utf8]{inputenc}
\usepackage{polski}
\usepackage{amsfonts}
\usepackage{amsmath,amssymb}

\DeclareMathOperator{\sech}{sech}
\DeclareMathOperator{\csch}{csch}
\DeclareMathOperator{\arcsec}{arcsec}
\DeclareMathOperator{\arccsc}{arcCsc}
\DeclareMathOperator{\arccosh}{arcCosh}
\DeclareMathOperator{\arcsinh}{arcsinh}
\DeclareMathOperator{\arctanh}{arctanh}
\DeclareMathOperator{\arcsech}{arcsech}
\DeclareMathOperator{\arccsch}{arcCsch}
\DeclareMathOperator{\arccoth}{arcCoth} 

%%%%%%%%%%%%%%%%%%%%%%%%%%%%%%%%%%%%%%%%%%%%%%%%%%%%%%%%%%%%%%%%%%%%%%%%%%%%%%%%
  \( \int \sqrt{x^2 + 1} \,dx\ =\ \frac12 \left( x\sqrt{x^2 + 1} + \text{arcsinh} x \right) + c\) 
  \( \int \sqrt{1 - x^2} \,dx\ =\ \frac12 \left( x\sqrt{1 - x^2} + \arcsin x \right) + c\) \\
  \( \int \frac{1}{ax^2 + bx + c} \,dx\ =\ \frac{2}{\sqrt{4ac - b^2}} \arctan
      \frac{2ax + b}{\sqrt{4ac - b^2}} \qquad \qquad (\Delta < 0)\) \\
  \( \int \frac{x}{ax^2 + bx + c} \,dx\ =\ \frac{1}{2a} \ln |ax^2 + bx + c|
      - \frac{b}{2a} \int \frac{\,dx}{ax^2 + bx + c} \) \\
  \( \int \tan x \,dx = - \ln |\cos x| + c \) \\
  \( (\arcsin x)' = \frac{1}{\sqrt{1 - x^2}}, \qquad \qquad
     (\arccos x)' = -\frac{1}{\sqrt{1 - x^2}} \) \\
  \( \mathtt{tgamma(t)} = \Gamma(t) = \int_0^{\infty} e^{t-1} e^{-x} \,dx \) \\
  \( \frac1{\pi} = 0.31831,\ \ \pi^2 = 9.86960,\ \ \frac1{\pi^2} = 0.10132,\ \ \frac1e = 0.36788,
  \ \ \gamma = 0.577215664901532\) \\
  \( H_n = \ln n + \gamma + \frac1{2n} - \frac1{12n^2} + O(n^{-4}) \) \\
  \( \ln n! = n \ln n - n + \frac12 \ln(2 \pi n) + \frac1{12n} - \frac1{360n^3} + \frac1{1260n^5}
  - O(n^{-7}) \)
  \\ 
  \\
	Jeśli $n=2^{a_0} \cdot {p_1}^{2 \cdot a_1} \cdot ... \cdot {p_r}^{2 \cdot a_r} \cdot {q_1}^{b_1} \cdot ... \cdot {q_s}^{b_s}$ i $B= \prod (b_i+1)$, gdzie $p_i$ to liczby postaci $4 \cdot k + 3$, a $q_i$ to liczby postaci $4 \cdot k + 1$, to liczba sposobów na zapisanie $n$ jako sumy dwóch kwadratów liczb naturalnych wynosi:
      \begin{itemize}
        \item 0, jeśli któraś z liczb $a_i$ nie jest całkowita
        \item $\frac{B}{2}$, jeśli $B$ jest parzyste
        \item $\frac{B-(-1)^{a_0}}{2}$, jeśli $B$ jest nieparzyste
      \end{itemize}
      
$S_j = \sum\limits_{1\leq i_1<i_2<\ldots<i_j\leq n} \mid A_{i_1}\cap A_{i_2} \cap \ldots A_{i_j} \mid$\\
$\sum\limits_{j\geq k}{j \choose k}(-1)^{j + k}S_j$ - liczba elementów należących do dokładnie $k$ zbiorów\\
$\sum\limits_{j\geq k}{j - 1 \choose k - 1}(-1)^{j + k}S_j$ - liczba elementów należących do co najmniej $k$ zbiorów.
$\sum\limits_{j\geq 1}(-2)^{j-1}S_j$ - liczba elementów należących do nieparzyście wiele zbiorów

Zaczynając w punkcie $0$, idąc w każdym ruchu o $-1$ z prawdopodobieństwem $p$ i o $+1$ z $1 - p$, kończąc w $-A$ lub $B$ prawdopodobieństwo dojścia do $B$ wynosi
$\frac{1 + r + r^2 + \ldots + r^{A - 1}}{1 + r + r^2 + \ldots + r^{A + B - 1}} = \frac{1 - r^A}{1 - r^{A + B}}^* = 1 - \frac{1 + r' + r'^2 + \ldots + r'^{B - 1}}{1 + r' + r'^2 + \ldots + r'^{A + B - 1}} = 1 - \frac{1 - r'^B}{1 - r'^{A + B}}*,$ \\ (*działa o ile $p \neq \frac12$) gdzie $r = \frac{p}{1 - p}, r' = \frac1r$, dla $p=\frac12$ równe $\frac{A}{A+B}$

Dla $x_1, x_2, \ldots, x_n$ macierz $a_{i,j}=x_i^{j - 1}$ ma wyznacznik $\prod\limits_{i<j}(x_j - x_i)$ i odwrotność
$b_{i,j} = (-1)^i\frac{\sum\limits_{A \subset [n]\setminus \lbrace j \rbrace, |A| = n - i}\prod\limits_{k \in A} x_k}{\prod\limits_{k\neq j}(x_k - x_j)}$
%%%%%%%%%%%%%%%%%%%%%%%%%%%%%%%%%%%%%%%%%%%%%%%%%%%%%%%%%%%%%%%%%%%%%%%%%%%%%%%%
